\documentclass[letter, 12pt]{article}
\usepackage[tmargin=.5in,lmargin=0.8in,rmargin=0.8in,bmargin=1in,paper=letterpaper]{geometry}
%\usepackage[mathlf,textlf,minionint]{minionpro}
\usepackage[LY1,OT1]{fontenc}
\usepackage{pifont,amssymb}
\usepackage{amsmath}
\usepackage[allowmove]{url}
\usepackage{fancyhdr,ifthen}
\usepackage{tikz}
\usepackage{multicol}
% --------------------------------------------------------------------
\newif\ifpdf \ifx\pdfoutput\undefined
\pdffalse
\else
\pdftrue
\fi
\usepackage[stretch=40,step=8,selected=true]{microtype}  % allow font expansion up to +/- 4% of normal width
% --------------------------------------------------------------------
\begin{document}
\newcommand{\smalltimes}{\mbox{\footnotesize $\times$}}
\newcommand{\isdef}{\stackrel{\mathrm{def}}{=}}
\newcommand{\eps}{\epsilon}
\newcommand{\ind}{\mbox{\hspace{0.25in}}}
\newcommand{\lh}{\stackrel{\mbox{\tiny{l'H}}}{=}}
\renewcommand{\Re}{\mathop{\rm {Re}}}
\renewcommand{\Im}{\mathop{\rm {Im}}}
\newcommand{\Res}{\mathop{\rm {Res}}}
\newcommand{\Arg}{\mathop{\rm {Arg}}}
\newcommand{\Log}{\mathop{\rm {Log}}}
\newcommand{\sech}{\mathop{\rm {sech}}}
\newcommand{\mat}[1]{\left(\begin{matrix}#1\end{matrix}\right)}
\newcommand{\erf}{\mathop{\rm {erf}}}
\newcommand{\erfc}{\mathop{\rm {erfc}}}
\newcommand{\vect}[1]{\vec{\mathbf{#1}}}
\renewcommand{\theenumi}{\arabic{enumi}}
\renewcommand{\labelenumi}{{\theenumi}.}
\renewcommand{\theenumii}{\alph{enumii}}
\renewcommand{\labelenumii}{(\theenumii)}
\renewcommand{\ttdefault}{cmtt}
\newcommand{\MyriadPro}[2]{\usefont{LY1}{MyriadPro}{#1}{#2}}
\parindent=0in
\thispagestyle{empty}
\pagestyle{empty}
\pagenumbering{arabic}
\setcounter{page}{0}

\mbox{ }

\vspace{-.2in}

\begin{center}
{\Large\usefont{LY1}{MyriadPro}{bl}{nw}Math 189r Fall 2016 Midterm}
\end{center}

\vspace{.3in}

\begin{large}
\noindent \textbf{Name:} \hrulefill \hrulefill \hrulefill 
\mbox{ }\textbf{Time start/end:} \hrulefill \hrulefill

\vspace{.3in}
Harvey Mudd College's Honor Code is in effect for all students taking this
exam.  If you feel unclear about any of the following instructions,
please ask for clarification.

\begin{itemize}
\item This exam should be completed in a contiguous \textbf{three hour} period.
    If you so desire, you may insert a \textbf{twenty minute break} at the
    \textbf{one an a half hour mark} to get food, etc., but you are still restricted
    to the resources available to you during the exam (no internet). You may not
    write on your exam during the break if you choose to take one.
\item No notes, books, computers or calculators will be allowed during
  the exam or break period except for one sheet of notes (8.5$\times$11'', front \&
  back) that you have prepared yourself.  When you are finished,
  staple the note sheet you used to the back of the exam.  Please turn in your exam paper 
  into the box outside Professor Gu's office (SHAN3420) no later than Monday, October, 6:30pm.
\item Points may be deducted for answers that are not explained clearly.
\item Please pace yourself as this exam has four questions, some with
  multiple parts.
\end{itemize}
\end{large}
\vspace{.1 in}
\begin{center} \begin{Large}
\begin{tabular}[b]{|l|l|} \hline
 Problem 1 & \hspace{.5 in} / 17 points \\ \hline
 Problem 2 & \hspace{.5 in} / 24 points \\ \hline
 Problem 3 & \hspace{.5 in} / 34 points \\ \hline
 Problem 4 & \hspace{.5 in} / 25 points \\ \hline
 Total     & \hspace{.5 in} / 100 points \\ \hline 
\end{tabular} \end{Large}
\end{center}

\newpage 

%%%%%%%%%%%%%%%%%%%%%%%%%%%%%%%%%%%%%%%%%%%%%%%%%%%%%%%%
%%                  Equation Sheet                    %%
%%%%%%%%%%%%%%%%%%%%%%%%%%%%%%%%%%%%%%%%%%%%%%%%%%%%%%%%

\begin{center}
    Potentially Useful Equations\\

    \[
        \mathbb{P}_{emp}(x) = \frac{1}{n}\sum_i \delta(x - x_i)
    \]
    \[
        \int \delta(x-t)f(x)dx = f(t)
    \]
    \[
        \|\mathbf{x}\|_1 = \sum_i |\mathbf{x}_i|
    \]
    \[
        \|\mathbf{x}\|_2^2 = \mathbf{x}^\top\mathbf{x}
    \]
    \[
        \mathrm{cov}[\mathbf{x}] = \mathbb{E}\left[(\mathbf{x} - \mathbb{E}[\mathbf{x}])(\mathbf{x} - \mathbb{E}[\mathbf{x}])^\top\right]
    \]
    \[
        \mathcal{N}(\mathbf{x}|\boldsymbol{\mu},\mathbf{\Sigma}) = \frac{1}{(2\pi)^{D/2}|\mathbf{\Sigma}|^{1/2}}\exp\left[-\frac{1}{2}(\mathbf{x} - \boldsymbol{\mu})^\top\mathbf{\Sigma}^{-1}(\mathbf{x} - \boldsymbol{\mu})\right]
    \]
    \[
        \mathbb{P}(\mathbf{x}_1|\mathbf{x}_2) = \mathcal{N}(x_1|\boldsymbol{\mu}_{1|2},\mathbf{\Sigma}_{1|2})
    \]
    \[
        \boldsymbol{\mu}_{1|2} = \boldsymbol{\mu}_1 + \mathbf{\Sigma}_{12}\mathbf{\Sigma}_{22}^{-1}(\mathbf{x}_2 - \boldsymbol{\mu}_2)
    \]
    \[
        \mathbf{\Sigma}_{1|2} = \mathbf{\Sigma}_{11} - \mathbf{\Sigma}_{12}\mathbf{\Sigma}_{22}^{-1}\mathbf{\Sigma}_{21}
    \]
    \[
        \hat{\boldsymbol{\mu}} = \frac{1}{N}\sum_i \mathbf{x}_i
    \]
    \[
        \hat{\mathbf{\Sigma}} = \sum_i (\mathbf{x}_i - \hat{\boldsymbol{\mu}})(\mathbf{x}_i - \hat{\boldsymbol{\mu}})^\top
    \]
    \[
        \mathrm{Cat}(\mathbf{x}|\theta) = \prod_{j=1}^K \theta_j^{\mathbb{I}(x_j = 1)}
    \]
    \[
        \mathbb{P}(\mathbf{x}|\theta) = \frac{1}{Z(\theta)}h(\mathbf{x})\exp\left(\theta^\top \phi(\mathbf{x})\right)
    \]
    \[
        \mathbb{KL}(p\|q) = \int p(x)\log\frac{p(x)}{q(x)}dx
    \]
    \[
        \nabla_\mathbf{x}\left(\mathbf{x}^\top A\mathbf{x} + \mathbf{b}^\top\mathbf{x}\right) = (A + A^\top)\mathbf{x} + \mathbf{b}
    \]
    \[
        \nabla^2_\mathbf{x}\left(\mathbf{x}^\top A\mathbf{x} + \mathbf{b}^\top\mathbf{x}\right) = (A + A^\top)
    \]
\end{center}
\newpage

%%%%%%%%%%%%%%%%%%%%%%%%%%%%%%%%%%%%%%%%%%%%%%%%%%%%%%%%
%%                     Problems                       %%
%%%%%%%%%%%%%%%%%%%%%%%%%%%%%%%%%%%%%%%%%%%%%%%%%%%%%%%%
\begin{enumerate}
\item (17 Points) Consider the Ridge Regression optimization problem
    \[
        \mathrm{minimize:}~f(\mathbf{x}) = \|\mathbf{A}\mathbf{x} - \mathbf{b}\|_2^2 + \|\mathbf{\Gamma}\mathbf{x}\|_2^2.
    \]
    \begin{enumerate}
        \item[(a)] Solve for the optimal $\mathbf{x}^\star$ in closed form.\vfill
        \item[(b)] Consider a dataset $\mathcal{D} = \{(0,1), (1,2),(2,1),(3,2)\}$.
            Construct a matrix $A$ and $\mathbf{b}$ from this dataset $\mathcal{D}$ and
            compute the Ridge estimate with $\mathbf{\Gamma} = \mathbf{I}$. Note you
            may leave your solution in the form of the inverse of a matrix times a
            vector instead of computing the inverse by hand. All other operations must
            be simplified.\vfill
    \end{enumerate}
    \clearpage

\item (24 Points - 3 each) Indicate whether each of the following
    statements is true (T) or not necessarily true (X).  Be careful with
    your answers as no partial credit will be awarded on this problem.
    \begin{enumerate}
    \item
      \raisebox{.1in}{\fbox{\parbox[t][.4in]{.4in}{\mbox{\hspace{.4in}}}}} \parbox[t]{5in}{
      The following is true: $\mathrm{cov}[A\mathbf{x} + \mathbf{b}] = A^\top\mathrm{cov}[\mathbf{x}]A$.
    }
    \vspace{.4in}
    \item
      \raisebox{.1in}{\fbox{\parbox[t][.4in]{.4in}{\mbox{\hspace{.4in}}}}} \parbox[t]{5in}{
      Consider minimizing $-\ell(\mathbf{w},\mathcal{D}_{train}) +
      \lambda\|\mathbf{w}\|_2^2$ where $\ell(\mathbf{w},\mathcal{D}) =
      \frac{1}{n}\sum_i y_i\log\sigma(\mathbf{w}^\top \mathbf{x}_i) + (1-y_i)\log(1 -
      \sigma(\mathbf{w}^\top\mathbf{x}_i))$ is the average
      log-likelihood on a dataset $\mathcal{D}$ for the $\ell_2$-regularized logistic
      regression model. If the training data is linearly separable, might some weights
      $w_j$ become infinite if $\lambda=0$? (Murphy 8.6.c)
    }
    \vspace{.4in}
    \item
      \raisebox{.1in}{\fbox{\parbox[t][.4in]{.4in}{\mbox{\hspace{.4in}}}}} \parbox[t]{5in}{
      Adding a zero-mean Gaussian Prior on each of the weights in a Logistic Regression
      model will encourage sparser weights than a zero-mean Laplace prior in a Maximum-a-Posteriori
      estimation of the optimal parameters.
      }
    \vspace{.4in}
    \item
      \raisebox{.1in}{\fbox{\parbox[t][.4in]{.4in}{\mbox{\hspace{.4in}}}}} \parbox[t]{5in}{
      Minimizing $\|X\mathbf{w} - \mathbf{y}\|_2^2$ and maximizing the likelihood
      $\prod_i \mathcal{N}(\mathbf{y}_i | \mathbf{w}^\top\mathbf{x}_i, \sigma^2)$ all
      result in the Normal Equations, the optimal solution $X^\top X\mathbf{w}^\star = X^\top \mathbf{y}$,
      in the context of linear regression.
      }
    \vspace{.4in}
    \item
      \raisebox{.1in}{\fbox{\parbox[t][.4in]{.4in}{\mbox{\hspace{.4in}}}}} \parbox[t]{5in}{
      A covariance matrix $\mathbf{\Sigma}$ could potentially have a negative eigenvalue.
      }
    \vspace{.4in}
    \item
      \raisebox{.1in}{\fbox{\parbox[t][.4in]{.4in}{\mbox{\hspace{.4in}}}}} \parbox[t]{5in}{
      The level sets of a multivariate Gaussian density are always ellipses of the form
      $\{\mathbf{x}: \mathbf{x}^\top A \mathbf{x} = k\}$.
      }
    \vspace{.4in}
    \item
      \raisebox{.1in}{\fbox{\parbox[t][.4in]{.4in}{\mbox{\hspace{.4in}}}}} \parbox[t]{5in}{
          The Multivariate Normal Distribution is not in the exponential family.
      }
    \vspace{.4in}
    \item
      \raisebox{.1in}{\fbox{\parbox[t][.4in]{.4in}{\mbox{\hspace{.4in}}}}} \parbox[t]{5in}{
          The goal of the Support Vector Machine is to maximize the margin, defined as the distance
          of the closest examples from the decision boundary.
      }

    \end{enumerate}
    \clearpage
\item (34 Points) Consider a Poisson distributed $X\sim \mathrm{Poi}(\lambda)$ defined
    over $X\in\{0,1,2,\dots\}$ with probability mass function
    \[
        \mathrm{Poi}(x|\lambda) = e^{-\lambda}\frac{\lambda^x}{x!}.
    \]
    Note that $\mathbb{E}[X] = \lambda$.
    \begin{enumerate}
        \item Show that the Poisson distribution is in the Exponential Family.\vfill\vfill
        \item Consider creating a Generalized Linear Model from the Poisson distribution
            to model some count data. This will work since we showed that the distribution
            is an exponential family distribution. What is the distribution of the
            predicted value $\hat y$ given a datapoint $\mathbf{x}_i$ and weights
            $\mathbf{w}$?\vfill\clearpage
        \item Derive an expression for the log-likelihood of a dataset $\mathcal{D}$,
            $\log\mathbb{P}(\mathcal{D}|\mathbf{w})$. Assume the data is identically and
            independently distributed.\vfill
        \item Suppose we place an isotropic ($\mathbf\Sigma = \sigma^2\mathbf{I}$)
            Gaussian prior on the weights $\mathbf{w}$. Derive an expression which, when
            maximized, would maximize
            $\log\mathbb{P}(\mathbf{w}|\mathcal{D})$ (basically ignore constants).
            \textit{Hint:} think about problem
            1.a from homework 1. You may use results from that part of the assignment.
            You may introduce an auxillary variable $\lambda$ which somehow relates
            to $\sigma^2$ (and you don't need to define that relationship exactly).
            \vfill
        \item Suppose we want to compute a maximum-a-posteriori estimate of $\mathbf{w}$
            given the prior from the previous problem. State the optimization problem
            we are trying to solve. Is this solvable in closed form? If so, solve it.
            Otherwise, compute the gradient of the objective function with respect
            to $\mathbf{w}$.\vfill
    \end{enumerate}
    \clearpage
\item (25 Points) Suppose we want to model where we have some input
    data $\mathbf{X}$, with each datapoint corresponding to some observed function output
    $\mathbf{f}$. We also have points $\mathbf{X}_\star$ with which we want to predict
    what the function output $\mathbf{f}_\star$ will be. Our modelling assumption is this:
    \begin{enumerate}
        \item[(1)] The function outputs are jointly normal with mean $\boldsymbol{\mu}$ and
            covariance $\mathbf{K}$ such that
            \[
                \mat{\mathbf{f}\\\mathbf{f}_\star} \sim \mathcal{N}\left(\mat{\boldsymbol{\mu}\\\boldsymbol{\mu}_\star},\mat{\mathbf{K}&\mathbf{K}_\star\\\mathbf{K}_\star^\top&\mathbf{K}_{\star\star}}\right)
            \]
        \item[(2)] The mean $\boldsymbol{\mu} = \boldsymbol{\mu}(X) =
            \mat{m(\mathbf{x}_i),\dots,m(\mathbf{x}_n)}$ for $m : \mathcal{X} \rightarrow
            \mathbb{R}$ where $\mathcal{X}$ is the space your data live in (points on
            the line, locations on earth, molecules, etc.). Notationally,
            $\boldsymbol{\mu}_\star = \boldsymbol{\mu}(X_\star)$.
        \item[(3)] The covariance between two datapoints $\mathbf{K}_{ij} =
            \kappa(\mathbf{x}_i,\mathbf{x}_j)$ where $\kappa : \mathcal{X}\times
            \mathcal{X}\rightarrow \mathbb{R}$. Later in the course we'll call this
            $\kappa(\cdot,\cdot)$ a valid kernel. This implies
            that $\mathbf{K}\in\mathbb{R}^{n\times n}$, $\mathbf{K}_\star\in
            \mathbb{R}^{n\times n_\star}$, and $\mathbf{K}_{\star\star}\in
            \mathbb{R}^{n_\star\times n_\star}$.
    \end{enumerate}
    \begin{enumerate}
        \item Compute the posterior predictive distribution
            $p(\mathbf{f}_\star|\mathbf{X}_\star,\mathbf{X},\mathbf{f})$ when
            we assume that the observed $\mathbf{f}$ has no noise. \textit{Hint:}
            consider conditioning a Gaussian.\vfill
        \item Compute the posterior predictive distribution
            $p(\mathbf{f}_\star|\mathbf{X}_\star,\mathbf{X},\mathbf{y})$ where
            we assume our observations $\mathbf{y} = f(\mathbf{x}) + \epsilon$ with
            $\epsilon
            \sim \mathcal{N}(0,\sigma^2)$. \textit{Hint:} this assumption will only
            change $\mathbf{f}$ to $\mathbf{y}$ and $\mathbf{K}$ (not $\mathbf{K}_\star$
            or $\mathbf{K}_{\star\star}$) into $\mathbf{K} + \sigma^2\mathbf{I}$
            from the equation in (1). For this part assume the mean $\boldsymbol{\mu}=
            \mathbf{0}$.\vfill
        \item Show that when predicting at only one location $\mathbf{x}_\star$ the
            mean of the predictive distribution from the previous part
            $\overline{\mathbf{f}} = \sum_{i=1}^n \alpha_i
            \kappa(\mathbf{x}_i,\mathbf{x}_\star)$ with $\alpha = (\mathbf{K} + \sigma^2
            \mathbf{I})^{-1}\mathbf{y}$ being a constant vector depending on your training
            data and kernel. Armed with this knowledge, how might be we restrict
            the predicted function $\overline{\mathbf{f}}(\mathbf{x})$ to be periodic?
            \textit{Hint:} note that $\kappa(\mathbf{x}_i,\mathbf{x}_\star)$ is function of
            $\mathbf{x}_\star$ and the sum of periodic functions is periodic.
            \vfill
    \end{enumerate}
    \clearpage
\end{enumerate}
\end{document}

