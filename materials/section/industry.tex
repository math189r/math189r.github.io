\documentclass[12pt,letterpaper]{article}
\usepackage[margin=1in]{geometry}
\usepackage{graphicx}
\usepackage{amsmath,amssymb,amsthm}
\usepackage{enumerate,mathtools}
\usepackage{algorithm2e}
\usepackage{tikz}
\usepackage{xcolor}
\usepackage{hyperref}

% Theorems
\usepackage{amsthm}
\renewcommand\qedsymbol{$\blacksquare$}
\makeatletter
\@ifclassloaded{article}{
    \newtheorem{definition}{Definition}[section]
    \newtheorem{example}{Example}[section]
    \newtheorem{theorem}{Theorem}[section]
    \newtheorem{corollary}{Corollary}[theorem]
    \newtheorem{lemma}{Lemma}[theorem]
}{
}
\makeatother

% Random Stuff
\setlength\unitlength{1mm}

\newcommand{\insertfig}[3]{
\begin{figure}[htbp]\begin{center}\begin{picture}(120,90)
\put(0,-5){\includegraphics[width=12cm,height=9cm,clip=]{#1.eps}}\end{picture}\end{center}
\caption{#2}\label{#3}\end{figure}}

\newcommand{\insertxfig}[4]{
\begin{figure}[htbp]
\begin{center}
\leavevmode \centerline{\resizebox{#4\textwidth}{!}{\input
#1.pstex_t}}
\caption{#2} \label{#3}
\end{center}
\end{figure}}

\long\def\comment#1{}

\newcommand\norm[1]{\left\lVert#1\right\rVert}
\DeclareMathOperator*{\argmin}{arg\,min}
\DeclareMathOperator*{\argmax}{arg\,max}

% bb font symbols
\newfont{\bbb}{msbm10 scaled 700}
\newcommand{\CCC}{\mbox{\bbb C}}

\newfont{\bbf}{msbm10 scaled 1100}
\newcommand{\CC}{\mbox{\bbf C}}
\newcommand{\PP}{\mbox{\bbf P}}
\newcommand{\RR}{\mbox{\bbf R}}
\newcommand{\QQ}{\mbox{\bbf Q}}
\newcommand{\ZZ}{\mbox{\bbf Z}}
\renewcommand{\SS}{\mbox{\bbf S}}
\newcommand{\FF}{\mbox{\bbf F}}
\newcommand{\GG}{\mbox{\bbf G}}
\newcommand{\EE}{\mbox{\bbf E}}
\newcommand{\NN}{\mbox{\bbf N}}
\newcommand{\KK}{\mbox{\bbf K}}
\newcommand{\KL}{\mbox{\bbf KL}}

% Vectors
\renewcommand{\aa}{{\bf a}}
\newcommand{\bb}{{\bf b}}
\newcommand{\cc}{{\bf c}}
\newcommand{\dd}{{\bf d}}
\newcommand{\ee}{{\bf e}}
\newcommand{\ff}{{\bf f}}
\renewcommand{\gg}{{\bf g}}
\newcommand{\hh}{{\bf h}}
\newcommand{\ii}{{\bf i}}
\newcommand{\jj}{{\bf j}}
\newcommand{\kk}{{\bf k}}
\renewcommand{\ll}{{\bf l}}
\newcommand{\mm}{{\bf m}}
\newcommand{\nn}{{\bf n}}
\newcommand{\oo}{{\bf o}}
\newcommand{\pp}{{\bf p}}
\newcommand{\qq}{{\bf q}}
\newcommand{\rr}{{\bf r}}
\renewcommand{\ss}{{\bf s}}
\renewcommand{\tt}{{\bf t}}
\newcommand{\uu}{{\bf u}}
\newcommand{\ww}{{\bf w}}
\newcommand{\vv}{{\bf v}}
\newcommand{\xx}{{\bf x}}
\newcommand{\yy}{{\bf y}}
\newcommand{\zz}{{\bf z}}
\newcommand{\0}{{\bf 0}}
\newcommand{\1}{{\bf 1}}

% Matrices
\newcommand{\Ab}{{\bf A}}
\newcommand{\Bb}{{\bf B}}
\newcommand{\Cb}{{\bf C}}
\newcommand{\Db}{{\bf D}}
\newcommand{\Eb}{{\bf E}}
\newcommand{\Fb}{{\bf F}}
\newcommand{\Gb}{{\bf G}}
\newcommand{\Hb}{{\bf H}}
\newcommand{\Ib}{{\bf I}}
\newcommand{\Jb}{{\bf J}}
\newcommand{\Kb}{{\bf K}}
\newcommand{\Lb}{{\bf L}}
\newcommand{\Mb}{{\bf M}}
\newcommand{\Nb}{{\bf N}}
\newcommand{\Ob}{{\bf O}}
\newcommand{\Pb}{{\bf P}}
\newcommand{\Qb}{{\bf Q}}
\newcommand{\Rb}{{\bf R}}
\newcommand{\Sb}{{\bf S}}
\newcommand{\Tb}{{\bf T}}
\newcommand{\Ub}{{\bf U}}
\newcommand{\Wb}{{\bf W}}
\newcommand{\Vb}{{\bf V}}
\newcommand{\Xb}{{\bf X}}
\newcommand{\Yb}{{\bf Y}}
\newcommand{\Zb}{{\bf Z}}

% Calligraphic
\newcommand{\Ac}{{\cal A}}
\newcommand{\Bc}{{\cal B}}
\newcommand{\Cc}{{\cal C}}
\newcommand{\Dc}{{\cal D}}
\newcommand{\Ec}{{\cal E}}
\newcommand{\Fc}{{\cal F}}
\newcommand{\Gc}{{\cal G}}
\newcommand{\Hc}{{\cal H}}
\newcommand{\Ic}{{\cal I}}
\newcommand{\Jc}{{\cal J}}
\newcommand{\Kc}{{\cal K}}
\newcommand{\Lc}{{\cal L}}
\newcommand{\Mc}{{\cal M}}
\newcommand{\Nc}{{\cal N}}
\newcommand{\Oc}{{\cal O}}
\newcommand{\Pc}{{\cal P}}
\newcommand{\Qc}{{\cal Q}}
\newcommand{\Rc}{{\cal R}}
\newcommand{\Sc}{{\cal S}}
\newcommand{\Tc}{{\cal T}}
\newcommand{\Uc}{{\cal U}}
\newcommand{\Wc}{{\cal W}}
\newcommand{\Vc}{{\cal V}}
\newcommand{\Xc}{{\cal X}}
\newcommand{\Yc}{{\cal Y}}
\newcommand{\Zc}{{\cal Z}}

% Bold greek letters
\newcommand{\alphab}{\hbox{\boldmath$\alpha$}}
\newcommand{\betab}{\hbox{\boldmath$\beta$}}
\newcommand{\gammab}{\hbox{\boldmath$\gamma$}}
\newcommand{\deltab}{\hbox{\boldmath$\delta$}}
\newcommand{\etab}{\hbox{\boldmath$\eta$}}
\newcommand{\lambdab}{\hbox{\boldmath$\lambda$}}
\newcommand{\epsilonb}{\hbox{\boldmath$\epsilon$}}
\newcommand{\nub}{\hbox{\boldmath$\nu$}}
\newcommand{\mub}{\hbox{\boldmath$\mu$}}
\newcommand{\zetab}{\hbox{\boldmath$\zeta$}}
\newcommand{\phib}{\hbox{\boldmath$\phi$}}
\newcommand{\psib}{\hbox{\boldmath$\psi$}}
\newcommand{\thetab}{\hbox{\boldmath$\theta$}}
\newcommand{\taub}{\hbox{\boldmath$\tau$}}
\newcommand{\omegab}{\hbox{\boldmath$\omega$}}
\newcommand{\xib}{\hbox{\boldmath$\xi$}}
\newcommand{\sigmab}{\hbox{\boldmath$\sigma$}}
\newcommand{\pib}{\hbox{\boldmath$\pi$}}
\newcommand{\rhob}{\hbox{\boldmath$\rho$}}

\newcommand{\Gammab}{\hbox{\boldmath$\Gamma$}}
\newcommand{\Lambdab}{\hbox{\boldmath$\Lambda$}}
\newcommand{\Deltab}{\hbox{\boldmath$\Delta$}}
\newcommand{\Sigmab}{\hbox{\boldmath$\Sigma$}}
\newcommand{\Phib}{\hbox{\boldmath$\Phi$}}
\newcommand{\Pib}{\hbox{\boldmath$\Pi$}}
\newcommand{\Psib}{\hbox{\boldmath$\Psi$}}
\newcommand{\Thetab}{\hbox{\boldmath$\Theta$}}
\newcommand{\Omegab}{\hbox{\boldmath$\Omega$}}
\newcommand{\Xib}{\hbox{\boldmath$\Xi$}}

% mixed symbols
\newcommand{\sinc}{{\hbox{sinc}}}
\newcommand{\diag}{{\hbox{diag}}}
\renewcommand{\det}{{\hbox{det}}}
\newcommand{\trace}{{\hbox{tr}}}
\newcommand{\tr}{\trace}
\newcommand{\sign}{{\hbox{sign}}}
\renewcommand{\arg}{{\hbox{arg}}}
\newcommand{\var}{{\hbox{var}}}
\newcommand{\cov}{{\hbox{cov}}}
\renewcommand{\Re}{{\rm Re}}
\renewcommand{\Im}{{\rm Im}}
\newcommand{\eqdef}{\stackrel{\Delta}{=}}
\newcommand{\defines}{{\,\,\stackrel{\scriptscriptstyle \bigtriangleup}{=}\,\,}}
\newcommand{\<}{\left\langle}
\renewcommand{\>}{\right\rangle}
\newcommand{\Psf}{{\sf P}}
\newcommand{\T}{\top}
\newcommand{\m}[1]{\begin{bmatrix} #1 \end{bmatrix}}


\begin{document}

\begin{center}
    \large Model Evaluation and Deployment;

    \large How Things Work in Industry

    \normalsize Conner DiPaolo
\end{center}
\vspace{3\baselineskip}

\section{Introduction}

This course is focuses on the mathematical approach to solving problems
involving a large amount of data, primarily using techniques from statistical
machine learning, but also building on non-statistical background such as
K-Nearest Neighbors or Support Vector Machines that reignited interest in
the field. That said, many people in the course have interest in moving to
industry to work with data there. This document is an eclectic collection
of techniques that, while not relevant to the learning goals in this course,
are must-haves for anyone looking to use machine learning techniques in industry
or even performing applied analysis in academia. Note that while these techniques
might speed up task completion in industry, following these are not necessarily
conducive to generating good research in academia.

\section{Model Complexity}

This is going to depend on where you are working, but unless your company is
doing more advanced work, everyone is likely to only use linear models in production.
A notable exception is that when working with images you're almost guaranteed to
use a convolutional neural network these days. This dichotomy between the set of models
that academia works on an industry works with (except neural networks) comes down
to a choice in efficiency. Complicated models are, in general
\begin{enumerate}[(i)]
    \item complicated to train (and therefore implement),
    \item might not have libraries available to train for you,
    \item non-trivial to store from a model-representation standpoint,
    \item and require more effort to maintain.
\end{enumerate}
The product of all of these issues brings us to a cardinal rule when working with
a task that seems to require machine learning: \textbf{try the simplest models that
    could possibly work first. If the simple model doesn't work well, try a more
complicated model only then.} This is a key departure from academic thinking. In industry,
the goal is to add features to a product. It follows that having $95\%$ accuracy (or some
other metric) is totally fine, even though using a more complicated model might let you
eek out a $2.5\%$ higher score on that metric. At it's core, time (to implement \textit{and}
maintain models) is prioritized higher than model performance. For a detailed and well
thought out analysis of the hidden costs of machine learning that motivate this intellectual
sparsity see Sculley et al.'s work, ``Machine Learning: The High-Interest Credit Card of
Technical Debt''\footnote{\url{http://static.googleusercontent.com/media/research.google.com/en//pubs/archive/43146.pdf}}
out of Google.

\section{Model Evaluation}

Usually the hardest part of using a machine learning method

\subsection{Cross Validation}

\section{Hyperparameter Optimization}

\section{Working With Text Data}

\end{document}
